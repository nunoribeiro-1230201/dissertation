% Chapter 1
% 
\chapter{Introduction} % Main chapter title
\label{chap:Chapter1} % For referencing the chapter elsewhere, use Chapter~\ref{Chapter1}


%-------------------------------------------------------------------------------
%---------
%
\section{Context and Problem} 

\section{Objectives}

\section{Ethical Considerations}

Ethical considerations play a crucial role in software engineering research, ensuring the integrity, transparency, and societal relevance of the work. This section outlines the ethical principles applied.

\textbf{Transparency and Fairness in Results.} As this research is focused on the comparison of programming languages, it is essential to maintain impartiality and avoid any bias in the results. Research integrity demands that results are not manipulated or altered to provoke more appealing discussions or gain community approval \cite{EthicDeb2018}. This dissertation adheres to the principle of transparency, ensuring that the benchmarking results reflect the true performance of each language.

\textbf{Replicability and Verification.} Replication is a important aspect of scientific research, enabling others to validate findings \footnote{NSPE Ethics code: \url{https://www.nspe.org/resources/ethics/code-ethics/} (accessed 1 December 2024)}. This dissertation involves the development of prototypes and proof-of-concepts to evaluate specific fault-tolerance strategies. To uphold ethical standards, all tests and configurations will be documented on a public repository to allow replication.

\textbf{Adherence to Professional Codes of Ethics.} This work adheres to the ethical principles outlined in the Institute of Electrical and Electronics Engineers (IEEE) Code of Ethics \footnote{IEEE Ethics code: \url{https://www.ieee.org/about/corporate/governance/p7-8.html/} (accessed 1 December 2024)} and the Association for Computing Machinery Code (AMC) of Ethics and Professional Conduct \footnote{ACM Ethics code: \url{https://www.acm.org/code-of-ethics/} (accessed 1 December 2024)}, which emphasize integrity, respect, fairness, and authorized use of content. Researchers must act responsibly by avoiding any practices that could harm the reputation or fairness of the comparison, maintaining an ethical commitment to the broader community of developers and researchers.

\textbf{Avoidance of Plagiarism and Proper Citation.} Plagiarism undermines the credibility and value of academic work. In alignment with the Code of Good Practices and Conduct of Polytechnic of Porto \footnote{IEEE Ethics code: \url{https://www.iscap.ipp.pt/regulamentos/CodigoboaspraticasedecondutaIPP.pdf/} (accessed 1 December 2024)}, particularly Article 10, this dissertation ensures proper attribution of all referenced works. Accurate citation is fundamental to acknowledge the contributions of others, demonstrate the research’s academic honesty, and respect intellectual property.

\section{Document structure} 